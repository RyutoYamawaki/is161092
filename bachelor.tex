\documentclass[a4j,9pt,twocolumn]{jsarticle}
\usepackage{apulayout}

\begin{document}
\mktitleb{卒業論文タイトル}
{****学科}{県大 太郎}{情報 花子}

\section{はじめに}
この文書では,愛知県立大学~情報科学部の卒業論文の要旨
(以降,要旨)の共通レイアウトの利用について説明しています.この文書にあ
る説明を参考にして要旨を作成して下さい.

\section{ドキュメントクラスの指定}
要旨は日本語で作成するので \LaTeX のドキュメントクラスは
\texttt{jsarticle} を使用します.オプションには,用紙サイズ \texttt{a4j},
フォントサイズは$9$,または$10\ {\rm pt}$を指定して下さい.
また,要旨は二段組で作成するので \texttt{twocolumn} も指定して下さい.
\begin{screen}
 \begin{small}
  \verb#\documentclass[a4j,9pt,twocolumn]{jsarticle}#
 \end{small}
\end{screen}

\section{レイアウト設定用ファイルの読み込み}
この文書と一緒に配布されている \texttt{apulayout.sty} をドキュメントクラ
スの指定直後に \verb#\usepackage# コマンドで読み込んで下さい.
\begin{screen}
 \begin{small}
  \verb#\documentclass[a4j,9pt,twocolumn]{jsarticle}#\\[-0.5em]
  \verb#\usepackage{apulayout}#
 \end{small}
\end{screen}

\texttt{apulayout.sty} には,要旨のレイアウトに必要な設定,コマンドが記
述されているので変更しないように注意して下さい.

設定用ファイル中で要旨の作成に必要な基本的なパッケージの読み込みは行っ
ています.読み込み済みのパッケージは \texttt{amsmath},\texttt{amssymb},
\texttt{ascmac},\texttt{graphicx} です.また,レイアウト用に
\texttt{fancyhdr},\texttt{nidanfloat},\texttt{geometry} パッケージの読
み込みも行っています.
このほかのパッケージが必要な場合には,\texttt{apulayout.sty} を読み込ん
だ後に,各自で読み込みを行って下さい.

\section{要旨タイトル部分}
要旨のタイトル部分は \verb#\mktitleb# コマンドを使用します.

\verb#\mktitleb# コマンドには $4$ つの引数があります.それぞれ,卒業論文
タイトル,学科名,学生氏名,指導教員氏名を指定します.学科名は各自の所属
する学科を情報システム学科,または地域情報科学科から選択して下さい.
\begin{screen}
 \begin{small}
  \verb#\begin{document}#\\[-0.5em]
  \verb#\mktitleb{卒業論文タイトル}{学科名}{学生氏名}{教員氏名}#
 \end{small}
\end{screen}

このコマンドは,
\texttt{document} 環境の開始直後に指定して下さい.

\section{二段組時の特殊な図,表の配置}
要旨は二段組で作成するため,横に長い図や表を \texttt{figure} や
\texttt{table} 環境を使用して配置すると A4 用紙一枚に収まらなくなります.
その場合には \texttt{figure*} や \texttt{table*} 環境を使用して下さい.
\figurename\ref{fig:exp} や \tablename\ref{tab:exp} のように要旨の下部に
一段組で配置されます.\\[-0.5em]

\begin{itembox}[l]{\texttt{figure*} 環境の記述例}
 \begin{small}
  \verb#\begin{figure*}#\\[-0.5em]
  \verb# \begin{center}#\\[-0.5em]
  \verb#  \includegraphics{fig.eps}#\\[-0.5em]
  \verb# \end{center}#\\[-0.5em]
  \verb#\end{figure*}#
 \end{small}
\end{itembox}

\begin{figure*}
 \begin{center}
  \includegraphics[width=0.8\textwidth]{figsample.eps}
  \vspace{-1em}
  \caption{二段抜きの例(\texttt{figure*}環境)}
  \label{fig:exp}
 \end{center}
\end{figure*}

\begin{table*}
 \caption{二段抜きの例(\texttt{table*} 環境)}
 \label{tab:exp}
 \vspace{-1em}
 \begin{center}
  \begin{tabular}{|c|c|}
   \hline
   要素 $1$&要素 $2$\\
   \hline
   二段組では表現がし難い横に長い表も&\texttt{table*} 環境を使えば二段抜きにすることができま
   す.\\
   \hline
  \end{tabular}
 \end{center}
\end{table*}

\section{文章サンプル}
After a seven-year journey, a NASA space capsule returned safely to
Earth on Sunday with the first dust ever fetched from a comet, a cosmic
bounty that scientists hope will yield clues to how the solar system
formed.

The capsule's blazing plunge through the atmosphere lit up parts of the
western sky as it capped a mission in which the Stardust spacecraft
swooped past a comet known as Wild 2.
\centerline{$\vdots$}\\[-0.5em]
\hfill{\cite{web2} より転載}
\section{おわりに}
この要旨作成用のレイアウト設定ファイルの動作確認は,Vine Linux 3.2 上の
\texttt{tetex-2.0.2-0vl14} がインストールされている環境で行いました.

この文書は \LaTeX を使用して要旨を作成するときに共通のレイアウトの利用に
ついてのみ説明しています.\LaTeX の基本的な操作やコマンド,環境について
は演習室で閲覧することのできる \cite{美} や \cite{典} を参照して下さい.
また,インターネットを利用して検索することも効果的です.

\begin{thebibliography}{9}
 \bibitem{美} 奥村晴彦 『\verb#[#改訂版\verb#]# \LaTeX2e 美文書作成入門』 技術
       評論社,2003
 \bibitem{典} 生田誠三 『\LaTeX2e 文典』 朝倉書店,2001
 \bibitem{web} \texttt{http://pcg-xr1g.hp.infoseek.co.jp/cat\_latex.shtml}
 \bibitem{web2} \texttt{http://www.chinadaily.com.cn/english/doc/\\2006-01/16/content\_512520.htm}
\end{thebibliography}
\end{document}
